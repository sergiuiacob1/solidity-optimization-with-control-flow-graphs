\chapter*{YUL Intermediate Representation} 
\addcontentsline{toc}{chapter}{YUL Intermediate Representation}


\section*{Introduction}
YUL IR is relatively new in the Solidity compiler.

First mention of YUL IR: 3rd of December, 2018, version 0.5.1.

Compiling via the YUL IR considered production ready with 0.8.13 release, March 16, 2022.
Solidity's team focus is now on the YUL IR and they encouraged the usage of `--optimize --via-ir` at the Solidity Summit, April 2022 – as mentioned by Hari Mulackal.

Source: https://blog.soliditylang.org/category/releases/


Why is YUL necessary?
* Figuring out optimizations on EVM bytecode is a hassle. Bytecode is not readable (lizibil)
* YUL IR has a readable syntax – "medium" level, between Solidity and EVM bytecode
* Generating tests is much easier – expected output vs actual output


https://docs.soliditylang.org/en/latest/yul.html#

Obiective
1. Programs written in Yul should be readable, even if the code is generated by a compiler from Solidity or another high-level language.

2. Control flow should be easy to understand to help in manual inspection, formal verification and optimization.

3. The translation from Yul to bytecode should be as straightforward as possible.

4. Yul should be suitable for whole-program optimization.


Proprietati
1. Yul is statically typed (to avoid confusion between vals / references for example)




Asta de adaugat in alta sectiune

Bytecode based optimizer (din slide-uri Hari)
> works on basic blocks \ref{cfg-basic-block} (de prezentat control flow graph-urile inainte de asta)
> cannot really perform more complex optimizations
    > De exemplu, https://www.youtube.com/watch?v=BWO7ij9sLuA&list=PLX8x7Zj6Vezl1lqBgxiQH3TFbRNZza8Fk&index=17&ab_channel=SoliditySummit
    > minutul 7:00, de transpus in cuvinte
    > variabila e OUTSIDE the basic block, pentru ca leader-ul basic block-ului e acel JUMPDEST
    > It could, but the engineering consens is that it should not – TOO dangerous!!!
> Optimizing on YUL is much, much easier <-- see Hari's Summit talk (2022), 1st of May



Caveats of Optimizing
> Inline is currently an heuristic.
> Inline is awesome, but it can cause problems. The EVM can access only the first 16 stack slots. If you inline too much, it causes stack too deep issues. Balance between --optimize-runs


Alte avantaje
> function inlining – huge gas advantage, much easier to do in YUL 